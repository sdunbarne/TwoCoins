%%% -*-LaTeX-*-
%%% twocoins.tex.orig
%%% Prettyprinted by texpretty lex version 0.02 [21-May-2001]
%%% on Mon Oct 12 11:15:19 2020
%%% for Steve Dunbar (sdunbar@family-desktop)

\documentclass[12pt]{article}

\input{../../../../etc/macros} %% \input{../../../../etc/mzlatex_macros}
\input{../../../../etc/pdf_macros}

\bibliographystyle{plain}

\begin{document}

\myheader \mytitle

\hr

\sectiontitle{Distinguishing A Biased Coin From a Fair Coin}

\hr

\usefirefox

\hr

% \visual{Study Tip}{../../../../CommonInformation/Lessons/studytip.png}
% \section*{Study Tip}

% \hr

\visual{Rating}{../../../../CommonInformation/Lessons/rating.png}
\section*{Rating} %one of
% Everyone: contains no mathematics.
% Student: contains scenes of mild algebra or calculus that may require guidance.
Mathematically Mature:  may contain mathematics beyond calculus with
proofs.  \\
Mathematicians Only:  prolonged scenes of intense rigor.

\hr

\visual{Section Starter Question}{../../../../CommonInformation/Lessons/question_mark.png}
\section*{Section Starter Question}

You have two identical coins except one is fair and the other has a
biased chance \( p > \frac{1}{2} \) of coming up heads.  Unfortunately,
you don't know which is which.  How can you find the biased coin?

\hr

\visual{Key Concepts}{../../../../CommonInformation/Lessons/keyconcepts.png}
\section*{Key Concepts}

\begin{enumerate}
    \item
        How to numerically compute the probability of a majority of
        heads in a sequence of paired coin flips of a biased coin and a
        fair coin.
    \item
        How to numerically compute the probability of statistical
        evidence of a biased coin in a sequence of paired coin flips of
        a biased coin and a fair coin.
    \item
        Bayesian analysis using likelihood ratios as statistical
        evidence of a biased coin in a sequence of paired coin flips of
        a biased coin and a fair coin.
    \item
        A theoretical analysis using probabilistic inequalities shows a
        necessary and sufficient number of flips is of the order \( (p-1/2)^
        {-2} \).
\end{enumerate}

\hr

\visual{Vocabulary}{../../../../CommonInformation/Lessons/vocabulary.png}
\section*{Vocabulary}
\begin{enumerate}
    \item
        A sequence of independent Bernoulli trials with probability \(
        1/2 \) of success on each trial is metaphorically called a \defn
        {fair} coin.  One for which the probability is not \( 1/2 \) is
        a \defn{biased} or \defn{unfair} coin.
    \item
        Suppose a fair coin \( 1 \) is in the left hand and a biased
        coin \( 2 \) is in the right hand.  The coin in the left hand
        comes up heads \( k_L \) times and the coin in the right hand
        comes up heads \( k_R \) times.  Let \( k = (k_L, k_R) \) and
        call this scenario \( \theta_L \).  The \defn{likelihood} of
        this scenario occurring is
        \[
            Q(k \given \theta_L) = B(k_L, n, p_1)B(k_R, n, p_2).
        \]
    \item
        If fair coin \( 1 \) is in the right hand and biased coin \( 2 \)
        is in the left hand then call this scenario \( \theta_R \).  The
        \defn{likelihood ratio} is
        \[
            \frac{Q(k \given \theta_L)}{Q(k \given \theta_R)}.
        \]
\end{enumerate}

\hr

\visual{Mathematical Ideas}{../../../../CommonInformation/Lessons/mathematicalideas.png}
\section*{Mathematical Ideas}

\subsection*{A Simple Probability Problem}

This problem serves as a warm-up for the more detailed investigations
below.

You have two coins.  One is fair with \( \Prob{H} = \frac{1}{2} \).  The
other coin is biased with \( \Prob{H} = \frac{2}{3} \).  First you toss
one of the coins once, resulting in heads.  Then you toss the other coin
three times, resulting in two heads.  Which coin is more likely to be
the biased coin, the first or the second?

\subsubsection*{Joint Probability Solution}

Assuming tossing the biased coin once and tossing the fair coin three
times, the probability of observing the outcome is
\[
    \frac{2}{3} \cdot \binom{3}{2} \left( \frac{1}{2} \right)^2 \left(
    \frac{1}{2} \right)^1 = \frac{2}{8}.
\]

On the other hand, assuming tossing the fair coin once and tossing the
biased coin three times, the probability of observing the given outcome
is
\[
    \frac{1}{2} \cdot \binom{3}{2} \left( \frac{2}{3} \right)^2 \left(
    \frac{1}{3} \right)^1 = \frac{2}{9}.
\]

This means it was more likely the biased coin was tossed once while the
fair coin was tossed three times.

\subsubsection*{Using Bayes Theorem}

Take a uniform prior, that is, assume the fair coin is equally likely to
be either of the two coins.  Let \( K = (k_1, k_2) \) denote the
observation, and \( B \) the event that the first coin is biased. Then
\[
    \Prob{B} = \Prob{B^C} = \frac{1}{2}.
\]

The conditional probabilities are
\[
    \Prob{K \given B} = \frac{2}{3} \cdot \binom{3}{2} \left( \frac{1}{2}
    \right)^2 \left( \frac{1}{2} \right)^1 = \frac{2}{8}
\] and
\[
    \Prob{K \given B^C} = \frac{1}{2} \cdot \binom{3}{2} \left( \frac{2}
    {3} \right)^2 \left( \frac{1}{3} \right)^1 = \frac{2}{9}.
\]

From Bayes Theorem,~%
\index{Bayes Theorem}
\begin{align*}
    \Prob{B \given K} &= \frac{\Prob{K \given B} \cdot \Prob{B}}{\Prob{K
    \given B} \cdot \Prob{B} + \Prob{K \given B^C} \cdot \Prob{B^C}} \\
    &= \frac{2/8}{2/8 + 2/9} = \frac{9}{17} > \frac{1}{2}.
\end{align*}

\subsubsection*{Using Likelihood Ratios}

Take even prior odds to be even, that is, assume the fair coin is
equally likely to be either of the two coins.  Let \( K = (k_1, k_2) \)
be the observed outcomes.  The likelihood ratio is~%
\index[likelihood ratio]

\begin{align*}
    L(K) &= \frac{\Prob{K \given B}}{\Prob{K \given B^C}} \\
    &= \frac{\frac{2}{3} \binom{3}{2} \left( \frac{1}{2} \right)^3}{\frac
    {1}{2} \cdot \binom{3}{2} \left( \frac{2}{3} \right)^2 \left(\frac{1}
    {3} \right)^1} \\
    &= \frac{6/24}{12/54} = \frac{9}{8}.
\end{align*}

Then the odds in favor of the first coin being biased and the second
coin being fair are \( 9 :  8 \), so the probability is \( \frac{9}{8+9}
= \frac{9}{17} \).

\subsection*{A Larger Problem}

The following problem appeared in the FiveThirtyEight.com weekly Riddler
puzzle column on September 29, 2017:

\begin{quotation}
    On the table in front of you are two coins.  They look and feel
    identical, but you know one of them has been doctored.  The fair
    coin comes up heads half the time while the doctored coin comes up
    heads \( 60 \) percent of the time.  How many flips -- you must flip
    both coins at once, one with each hand -- would you need to give
    yourself a \( 95 \) percent chance of correctly identifying the
    doctored coin?

    Extra credit:  What if, instead of \( 60 \) percent, the doctored
    coin came up heads some percent \( p \) of the time?  How does that
    affect the speed with which you can correctly detect it?
\end{quotation}

This problem appeared in a paper ``What's Past is \emph{Not} Prologue''
by James White, Jeff Rosenbluth, and Victor Haghani
\cite{white17}.  In turn, a problem posed in a paper ``Good and bad
properties of the Kelly criterion'' by MacLean, Thorp and Ziemba
\cite{maclean10} inspired the problem by White, Rosenbluth, and Haghani.

Solving this problem requires some interpretation and computation.

\subsection*{Probability of a Majority of Heads}

A sequence of independent Bernoulli trials with probability \( 1/2 \) of
success on each trial is metaphorically called a \defn {fair} coin. One
for which the probability is not \( 1/2 \) is a \defn{biased} or \defn{unfair}
coin.  Continuing the coin metaphor, The trials are often called flips.

Any fixed number of simultaneous flips always has a chance that the fair
coin will have more heads than the biased coin.  But the Weak Law of
Large Numbers says the probability that the biased coin will have a
majority of heads increases to \( 1 \) as the number of flips increases.
One way to decide which is the biased coin is to choose the coin which
has a majority of heads.  The authors White, Rosenbluth, and Haghani
want to calculate how many paired flips of \( 2 \) coins, one biased and
one fair, we must observe in order to be \( 95\% \) confident that the
coin with more heads is the biased coin.

Assuming independence, since each coin has a binomial distribution the
joint probability mass distribution after \( n \) flips is the product
of the individual binomial probability mass distributions.%
\index{binomial distribution}
Denote by \( Q(n; j,k) \) the probability of \( j \) heads for the fair
coin and \( k \) heads for the biased in \( n \) flips.  Use \( p \) for
the probability of heads of the biased coin and \( \frac{1}{2} \) for
the probability of heads for the fair coin.  Then
\[
    Q( n; j, k) = \binom{n}{j} \left( \frac{1}{2} \right)^j \left( \frac
    {1}{2} \right)^{n-j} \cdot \binom{n}{k} p^k (1-p)^{n-k}.
\]%
\index{bivariate binomial distribution}

Then summing over values where \( j < k \) gives the probability that
the biased coin has more heads than the fair coin in \( n \) flips
\[
    \Prob{j<k} = \frac{1}{2^{n}} \sum\limits_{k \le n} \sum\limits_{j <
    k } \binom{n}{j} \binom{n}{k} p^k (1-p)^{n-k}.
\] This sum has no closed formula evaluation so calculation is
necessary.  First create two vectors of length \( n+1 \) with the
binomial distribution on \( 0 \) to \( n \) with probability \( \frac{1}
{2} \) for the fair coin and with probability \( p \) for the biased
coin.  Then using the outer product of these two vectors create the
bivariate binomial distribution for the two coins.  This will be an \( (n+1)
\times (n+1) \) matrix.  The element in row \( i \) and column \( j \)
is the product of the binomial probability of \( i-1 \) heads for the
fair coin and the binomial probability of \( j-1 \) heads for the biased
coin.  The sum is over column indices strictly greater than the row
indices.  To create the sum, set the lower triangular part of the matrix
and the diagonal of the matrix to \( 0 \) and use the sum command to sum
all entries of the matrix.  This seems to be efficient, even for values
of \( n \) up to \( 200 \).  To find the minimal value of \( n \) for
which this probability is greater than \( 0.95 \) use binary search in \(
n \) over a reasonable interval.

It might seem possible to use a bivariate normal approximation of the
bivariate binomial distribution to calculate the probability.  Although
approximation of the bivariate binomial distribution with a bivariate
normal distribution is possible, the double integration of the bivariate
normal with a non-zero covariance would be over a region of the form \(
y > x-\epsilon \).  The R language has no direct way to calculate the
integral over a region of this form, so it is actually easier to
calculate with the bivariate binomial distribution.

The result is that it takes \( 143 \) flips of the two coins for the
probability to be greater than \( 0.95 \) for the \( 0.6 \) biased coin
to have more heads than the fair coin.

The same analysis for various probabilities of heads \( p \) for the
biased coin and for values \( 0.95 \), \( 0.9 \) and \( 0.75 \) of
certainty is in Figure~%
\ref{fig:twocoins:majorityBinarySearch}.  The number of flips required
decreases as the bias \( p \) increases, as expected.  The number of
flips required also decreases as the certainty decreases.

\begin{figure}[htbp]
    \centerline{\includegraphics[]{majorityBinarySearch.pdf}}
    \caption{The number of flips $ N $ required to get a majority of
    heads with certainty $ 0.95 $ (black), $ 0.9 $ (red) and $ 0.75
    $ (blue).}%
    \label{fig:twocoins:majorityBinarySearch}
\end{figure}

\subsection*{Using the Central Limit Theorem}

Assume the biased coin is in the left hand.%
\index{biased coin}
Let \( L_j \) be the result of left-hand coin flip \( j \), knowing it
is biased so \( L_j = \text{Head} = 1 \) with probability \( 0.6 \), \(
L_j = \text{Tail} = 0 \) with probability \( 0.4 \).  Let \( R_j \) be
the result of right-hand coin flip \( j \), knowing it is fair so \( R_j
= \text{Head} = 1 \) with probability \( 0.5 \), \( R_j = \text{Tail} =
0 \) with probability \( 0.5 \).  Let \( X_j = L_j - R_j \).  This is a
trinomial random variable with \( X_j = -1 \) with probability \( 1/5 =
0.2 \), \( X_j = 0 \) with probability \( 1/2 = 0.5 \), \( X_j = 1 \)
with probability \( 3/10 = 0.3 \).

Consider the statistics of \( X_j \),
\begin{align*}
    \E{X_j} &= (-1)\cdot(0.2) + 0 \cdot (0.5) + (+1)\cdot(0.3) = 0.1, \\
    \Var{X_j} &= (-1)^2\cdot(0.2)+(0)^2\cdot(0.5)+(+1)^2\cdot(0.3)-(0.1)^2\\
    & =0.49, \\
    \sigma[X_j] &= 0.7.  \\
\end{align*}

Let \( S_n = \frac{1}{n} \sum\limits_{j=1}^n X_j \) be the sample mean.
The distribution of \( S_n \) is on \( -1 \) to \( 1 \) by increments of
\( 1/n \) for a total of \( 2n+1 \) points with
\begin{align*}
    \E{S_n} &= 0.1, \\
    \Var{S_n} &= 0.49/n, \text{ by independence, } \\
    \sigma[S_n] &= 0.7/\sqrt{n}.
\end{align*}
Thus the distribution of \( S_n \) clusters around \( 1/10 \) with
standard deviation \( 7/(10 \sqrt{n}) \).%
\index{sample mean|distribution}
If the biased coin is in the right hand, the distribution of \( S_n \)
clusters around \( -1/10 \) with standard deviation \( 7/(10 \sqrt{n}) \).%
The goal is find a number of flips such that with high probability the
sample mean \( S_n \) is closer to the theoretical mean \( 0.1 \) (when
the biased coin is in the left hand) than the value \( -0.1 \) (the
theoretical mean coming from the situation with the biased coin in the
right hand).  Take ``the sample mean closer to \( 0.1 \) than \( -0.1 \)''
to mean \( S_n \) is closer to \( 0.1 \) than the distance to the
midpoint \( 0 \) between the two means.  Precisely, the goal is to find \(
n \) so that \( \Prob{S_n - 0.1 > -0.1} \ge 0.95 \).  This is the same
event as \( \Prob{S_n > 0} = \Prob{L_n-R_n > 0} = \Prob{L_n > R_n} \)
which is the same criterion as having the majority of heads above.

Expressing the precise distribution of \( S_n \) analytically is
difficult.  So instead, use a numerical calculation of the distribution.
To numerically calculate the distribution of the sample mean \( S_{n} \),
use the R package \texttt{distr} and specifically the function \texttt{convpow}
that takes the \( n \)-fold convolution power of a distribution to
create the distribution of the \( n \)-fold sum of a random variable.%
\index{convolution power}
Note that mathematically the support of the distribution of, for
instance \( S_{100} \), would be from \( -1 \) to \( 1 \), with \( 201 \)
points.  However, the actual calculated distribution support of, for
instance, \( S_{100} \), is \( 111 \) points from \( -46 \) to \( 64 \).
The reason is that \texttt{convpow} ignores points with probability less
than \( 10^{-16} \) and so these points are not included in the domain.
So use \texttt{match(0, (support(D100)))} to find the index of \( 0 \).
This turns out to be index \( 47 \).  So summing the distribution over
indices from \( 47+1 \) to \( 111 \) gives the probability of the random
variable \( S_{n} > 0 \).  Searching for a value of \( n \) large enough
that the probability exceeds \( 0.95 \) gives the required number of
flips.  From some experimentation, the numerical support of the
distribution is positive for \( n \ge 150 \), that is, \( \Prob{S_{n}
\ge 0} = 1 \) for \( n \ge 150 \).  That means that the numerical search
should start from a high of \( 150 \).

Using this algorithm, the necessary number of flips to distinguish a
biased coin with a probability of heads \( 0.6 \) from a fair coin with
a certainty level of \( 0.95 \) is \( 143 \).  The same analysis for
various probabilities of heads \( p \) for the biased coin and for
values \( 0.95 \), \( 0.9 \) and \( 0.75 \) of certainty is in Figure~%
\ref{fig:twocoins:statisticalBinarySearch}.  The number of flips
required decreases as the bias increases as expected.  The number of
flips required also decreases to \( 5 \) for certainty \( 0.95 \), \( 4 \)
for certainty \( 0.9 \) and \( 3 \) for certainty \( 0.75 \).

\begin{figure}[htbp]
    \centerline{\includegraphics[]{statisticalBinarySearch.pdf}}
    \caption{The number of flips $ N $ required to statistically
    distinguish a coin with probability $ p $ from a fair coin with
    $ 0.95 $ (black), $ 0.9 $ (red) and $ 0.75 $ (blue).}%
    \label{fig:twocoins:statisticalBinarySearch}
\end{figure}

\subsection*{Connections between the Two Calculations}

\begin{figure}[htbp]
    % \centerline{\includegraphics[]{bivariate_binomial.pdf}}
\begin{asy}
settings.outformat = "pdf";

import graph;

size(5inches);

real myfontsize = 12;
real mylineskip = 1.2*myfontsize;
pen mypen = fontsize(myfontsize, mylineskip);
defaultpen(mypen);

int N  = 10;

filldraw( (1,10)--(10,10)--(10,1)--cycle, lightgrey);

for (int i=0; i <= N; ++i) {
  for (int j=0; j <= N; ++j) {
    dot((i,j));
  }
}

xaxis(L=Label("biased heads", align=(0,1)), xmin=0, xmax=10, axis=YEquals(N+1), ticks=RightTicks(n=1));
yaxis(L="fair heads", ymin=0, ymax=10, axis=XEquals(-1), ticks=RightTicks(new  string(real x) {return string(N-x);}, n=1));

for(int i=0; i<=N; ++i) {
  draw((0,i)--(i+1, -1), red, Arrow);
  label(string(i-10), (i+1,-1), align = SE);
}

for(int i=1; i<=N; ++i) {
  draw((i,10)--(11, i-1 ), red, Arrow);
  label(string(i), (11,i-1), align = SE);
}
\end{asy}

    \caption[]{The support of the bivariate binomial and the calculation
    of the multinomial distribution for $ n = 10 $.}%
    \label{fig:twocoins:bivariate}
\end{figure}

The calculation of the probabilities uses fundamentally the same
information, as the diagram in Figure~%
\ref{fig:twocoins:bivariate} for the case \( n = 10 \) illustrates.  The
\( 11 \times 11 \) array of dots represents the support of the \emph{bivariate
binomial distribution} as a matrix, with rows from \( 0 \) to \( 11 \)
for the number of heads from the fair coin and columns \( 0 \) to \( 11 \)
for number of heads for the biased coin.

The probability that the majority of flips is from the biased coin is
the sum of the probabilities in the strict upper triangle of the support
of the bivariate distribution.  The shaded part of Figure~%
\ref{fig:twocoins:bivariate} shows this domain.

The \emph{multinomial probability distribution}%
\index{multinomial probability}
on \( -10 \) to \( 10 \) of the sum \( \sum\limits_{j=0}^{10} (L_j-R_j) \)
is the sum of the probabilities along diagonals of the bivariate
binomial distribution as indicated by the red arrows.  In the case the
biased coin has \( p=0.6 \) so the mean of \( L_n - R_n = 0.1 \), the
probability of the event
\begin{align*}
    [S_n > 0] &= \left[\frac{1}{n} \sum\limits_{j=1}^n (L_j-R_j) > 0
    \right] \\
    &= \left[\sum\limits_{j=1}^n (L_j-R_j) > 0 \right] \\
    &= \left[\sum\limits_{j=1}^n L_j > \sum\limits_{j=1}^n R_j \right]
\end{align*}
is the sum of the probabilities in the strict upper triangle.  This is
the same criterion as for the majority of heads above.  For \( p = 0.6 \)
the search using the sample mean again requires \( 143 \) pair flips to
distinguish the coins with certainty \( 0.95 \).

The framing of the application motivates the choice of criterion.  The
original problem posed in FiveThirtyEight.com comes from a white paper
by James White, Jeff Rosenbluth, and Victor Haghani from Elm Partners
Investing.  The paper ``Good and bad properties of the Kelly criterion''
by McLean, Thorp and Ziemba motivated their example.  The question is a
simplified and idealized version of an investment question about how to
identify a higher performing investment, modeled by a biased coin, from
a lower performing investment with just an even chance at making a
profit, modeled by a fair coin.  In that case, the \( 95\% \) certainty
of a majority of gains is a reasonable choice of criterion.  If the
question is merely to identify the biased coin, then it makes sense to
use the Central Limit Theorem to distinguish the mean given that the
biased coin is in the left hand from the mean given that the biased coin
is in the right hand.

\subsection*{Bayesian Solution}

The coin flips are independent.  So if a coin comes up heads with
probability \( p \) and we flip it \( n \) times, the binomial
distribution \( B(k, n, p) = \binom{n}{k} p^k(1-p)^{n-k} \) gives the
probability that it comes up heads exactly \( k \) times.  Number the
coins \( 1 \) and \( 2 \) probabilities of coming up heads \( p_{1} \)
and \( p_{2} \) respectively.  Flip each coin \( n \) times and record
how many times each coin come up heads, but we don't know which coin is
which.  Say the coins come up heads \( k_{L} \) and \( k_{R} \) times
for the left and right coin respectively.  Call the vector of observed
data \( k = (k_L, k_R) \).

Exactly two possible scenarios occur:
\begin{description}
    \item[Fair on Left]
        Fair coin \( 1 \) was in the left hand and biased coin \( 2 \)
        was in the right hand and the observed outcome is \( k = (k_L, k_R)
        \).  Call this scenario \( \theta_L \).  The \defn{likelihood}
        of this scenario occurring is
        \[
            Q(k \given \theta_L) = B(k_L, n, p_1)B(k_R, n, p_2).
        \]
    \item[Fair on Right]
        Fair coin \( 1 \) was in the right hand and biased coin \( 2 \)
        was in the left hand and the observed outcome is \( k = (k_L, k_R)
        \).  Call this scenario \( \theta_R \).  The likelihood of this
        scenario occurring is
        \[
            Q(k \given \theta_R) = B(k_L, n, p_2)B(k_R, n, p_1).
        \]
\end{description}
If fair coin \( 1 \) is in the right hand and biased coin \( 2 \) is in
the left hand then call this scenario \( \theta_R \).  The \defn{likelihood
ratio} is
\[
    \frac{Q(k \given \theta_L)}{Q(k \given \theta_R)}.
\]

Compute the posterior probability by using Bayes’ rule:
\[
    Q( \theta_L \given k ) = \frac{Q(k \given \theta_L) \cdot \Prob{\theta_L}}
    {Q(k \given \theta_L ) \cdot \Prob{\theta_L} + Q(k \given \theta_R )
    \cdot \Prob{\theta_R}}
\] and similarly for \( Q( \theta_R \given k ) \).%
\index{Bayes' rule}
Assume scenarios \( \theta_L \) and \( \theta_R \) have the same prior
probability so \( \Prob{\theta_L} = \Prob{\theta_R} = 1/2 \). Therefore,
the Bayesian condition for confidence simplifies to
\[
    \frac{Q(k \given \theta_L)}{Q(k \given \theta_L) + Q(k \given \theta_R)
    } > 1- \alpha \text{ or } \frac{Q(k \given \theta_R)}{Q(k \given
    \theta_L ) + Q(k \given \theta_R) } > 1- \alpha.
\]

To be confident with level \( \alpha \) about one of the scenarios then
the posterior probability must be greater than \( 1 - \alpha \).
\begin{description}
    \item[Fair on Left]
        If \( Q(\theta_L \given k)>1 -\alpha \), then we are confident
        in Scenario Fair on Left.
    \item[Fair on Right]
        If \( Q(\theta_R \given k)>1 - \alpha \), then we are confident
        in Scenario Fair on Right.
\end{description}
By definition, \( Q(\theta_L \given k ) + Q(\theta_R \given k ) = 1 \)
for all \( k \).  So the two cases above can't simultaneously be true
when \( \alpha < 1/2 \).  Rearranging these inequalities, we have the
likelihood ratio
\[
    \frac{Q(k \given \theta_L)}{Q(k \given \theta_R)} > \frac{1-\alpha}{\alpha}
    \text{ or } \frac{Q(k \given \theta_R)}{Q(k \given \theta_L)} >
    \frac{1-\alpha}{\alpha}.
\]%
\index{likelihood ratio}

The implication is that we can stop flipping coins once the difference
between log-likelihoods grows sufficiently large.  The smaller is \(
\alpha \), the larger the difference in log-likelihoods must be before
we are confident.  Simplify the expression by substituting the
definition for the likelihoods \( Q(k \given \theta_L) \) and \( Q(k
\given \theta_R) \) in terms of the binomial probabilities becoming
\[
    \frac{Q(k \given \theta_L)}{Q(k \given \theta_R)} = \left(\frac{p_1}
    {p_2} \right)^{k_L - k_R} \cdot \left(\frac{1-p_2}{1-p_1} \right)^{k_L
    - k_R}.
\] Take logarithms of both sides and combine the expressions into a
single inequality involving the absolute difference of log-likelihoods
\[
    \abs{\log Q(k \given \theta_L) - \log Q(k \given \theta_R)} > \log
    \left( \frac{1}{\alpha} - 1 \right).
\]%
\index{log-likelihood}
The implication is that we can stop flipping coins once the difference
between log-likelihoods grows sufficiently large.  The smaller is \(
\alpha \), the larger the difference in log-likelihoods must be before
we can declare that we are confident.  After simplification, this
becomes
\[
    \abs{k_L - k_R} \cdot \abs{ \log\left( \frac{1}{p_1} - 1 \right) -
    \log\left( \frac{1}{p_2} - 1 \right) } > \log \left( \frac{1}{\alpha}
    - 1 \right).
\]

The number of heads \( k_L \) and \( k_R \) in the left and right hands
are random variables.  In one hand the number of heads from the coin
with \( p_1 \) will have the binomial probability distribution with mean
\( np_1 \) and standard deviation \( \sqrt{np_1(1-p_1)} \).  In the
other hand the number of heads from the coin with \( p_1 \) will have
the binomial probability distribution with mean \( np_2 \) and standard
deviation \( \sqrt{np_2(1-p_2)} \).  Normal distributions with
corresponding means and standard deviations can approximate each
binomial distribution.

As a first approximation, substitute the empirical probabilities \( \hat
{p}_1 = k_L/n \) and \( \hat{p}_2 = k_R/n \) obtained from the
distribution means.  Rearrange to isolate \( n \) to obtain
\[
    n > \frac{1}{\abs{\hat{p}_1 - \hat{p}_2}} \cdot \frac{\log \left(\frac
    {1}{\alpha} - 1 \right)}{\abs{ \log\left( \frac{1}{p_1} - 1 \right)
    - \log\left( \frac{1}{p_2} - 1 \right) }}.
\] This expression is a lower bound on the number of samples required,
in terms of the confidence \( \alpha \), the known probabilities \( p_1 \)
and \( p_2 \) and the empirical probabilities \( \hat{p}_1 \) and \(
\hat{p}_2 \).  Note that if \( p_1 - p_2 \to 0 \), then \( n \to \infty \)
which is expected.  Taking \( \hat{p}_1 = p_1 = 0.5 \), \( \hat{p}_2 = p_2
= 0.6 \) and \( \alpha = 0.05 \), then \( n = 72.619 \).  More
generally, Figure~%
\ref{fig:twocoins:flips} shows a plot of the number of flips required as
a function of \( p_2 \) for fixed \( p_1 = 0.5 \) and fixed \( \alpha =
0.05 \), so in percentages, \( 95\% \) confidence.

\begin{figure}
    \centering
\begin{asy}
    import graph;

size(5inches, IgnoreAspect);

real myfontsize = 12;
real mylineskip = 1.2*myfontsize;
pen mypen = fontsize(myfontsize, mylineskip);
defaultpen(mypen);

real p1 = 0.5;
real alpha = 0.05;

real flips(real p2) {
  real topa = log( 1/alpha - 1);
  real diff = p2 - p1;
  real botp1 = log( 1/p1 - 1);
  real botp2 = log( 1/p2 - 1);
  return (1/abs(diff))*(topa/(abs(botp1 - botp2)));
}


real log10up(real x) {return log10(x);}
real pow10up(real x) {return pow10(x);}

scaleT yscale=scaleT(log10up,pow10up,logarithmic=true);
scale(Linear,yscale);

draw(graph(flips,0.51,0.98));
draw(graph(flips,0.01,0.49));

yaxis("Flips",ymin=1,ymax=flips(0.51),RightTicks(Label(Fill(white))),EndArrow);
xaxis("$p2$", xmin=0.0, xmax=1.0, RightTicks);
xequals(0.5, red+dashed);
\end{asy}
    \caption{Number of flips required as a function of $ p_2 $ for
    fixed $ p_1 = 0.5 $ and $ \alpha = 0.05 $.}%
    \label{fig:twocoins:flips}
\end{figure}

However, this is only a lower bound on the real number of flips required
to find the biased coin since \( k_L \) and \( k_R \) can be closer than
what was used above.  In fact, in the scenario Fair on Left can have \(
k_L \) in the range \( [0, np_1+z_{\beta}\sqrt{np_1(1-p_1)}] \) and \( k_R
\) in the range \( [ np_2 - z_{\beta}\sqrt{np_2(1-p_2)}, n] \).  The
probability of each of these events is determined by quantiles \( z_{\beta}
\) of the binomial distribution, or approximately by the normal
distribution.  Since \( \sqrt{np_2(1-p_2)} < \sqrt{n \cdot \frac{1}{2}
\cdot \frac{1}{2}} = \frac{1}{2}\sqrt{n} \)
\begin{align*}
    \abs{k_L - k_R} &> (np_2 - z_{\beta}\sqrt{np_2(1-p_2)}) - (np_1+z_{\beta}\sqrt
    {np_1(1-p_1)}) \\
    &= n(p_2-p_1) - z_{\beta}(\sqrt{p_2(1-p_2)} + \sqrt{p_1(1-p_1)})
    \sqrt{n} \\
    &> n(p_2-p_1) - z_{\beta} \sqrt{n}
\end{align*}
with probability determined by the quantile \( z_{\beta} \).  Note that \(
p_2-p_1 >0 \), so \( n(p_2-p_1) - z_{\beta} \sqrt{n} \) is positive for
sufficiently large \( n \). The number of flips then required to
identify the biased coin is then determined by
\[
    n(p_2-p_1) - z_{\beta} \sqrt{n} > \frac{\log \left(\frac{1}{\alpha}
    - 1 \right)}{\abs{ \log\left ( \frac{1}{p_1} - 1 \right) - \log\left
    ( \frac{1}{p_2} - 1 \right) }}.
\] Choosing a quantile \( z_{\beta} \), substituting for \( p_1 \), \( p_2
\) and \( \alpha \), and then solving for \( n \) gives a lower bound
for the number of flips required for distinguishing the fair coin from
the biased coin.

For example, choosing \( z_{\beta} = 1.96 \) gives the approximate
probability \( \beta = 0.975 \) that \( k_L \) is in the range \( [0, np_1+z_
{\beta}\sqrt{np_1(1-p_1)}] \) and independently that \( k_R \) is in the
range \( [ np_2 - z_{\beta}\sqrt{np_2(1-p_2)}, n] \).  Then the
probability of both events is \( (0.975)^2 \approx 0.95 \). Using \( p_1
= 0.5 \), \( p_2 = 0.6 \) and \( \alpha = 0.05 \), the inequality is
approximately
\[
    0.1 n - 1.96 \sqrt{n} > 7.26.
\] Solving for \( n \) gives \( n \ 519.20 \).  As another example set \(
z_{\beta} = 0.559 \) with an approximate probability of \( 0.712 \) of \(
k_L \) in the range \( [0, np_1+z_{\beta}\sqrt{np_1(1-p_1)}] \) and
independently that \( k_R \) in the range \( [ np_2 - z_{\beta}\sqrt{np_2
(1-p_2)}, n] \) and joint probability of \( 0.507 \).  Then solving \(
0.1 n - 0.559 \sqrt{n} > 7.26 \) gives \( n = 138.35 \).  As a last
example, choosing \( z_{\beta} = 0 \) gives a probability of \( 0.5 \)
of \( k_L \) in the range \( [0, np_1] \) and independently with
probability approximately \( 0.5 \) that \( k_R \) in the range \( [ np_2,
n] \) and joint probability of \( 0.25 \).  This is equivalent to the
original calculation giving the lower bound of \( n=72.6 \).

The extended Bayesian analysis introduces another parameter \( z_{\beta}
\).  Using two parameters \( z_{\beta_1} \) and \( z_{\beta_2} \), as
long as \( np_1+z_{\beta_1}\sqrt{np_1(1-p_1)} < np_2-z_{\beta_2}\sqrt{np_2
(1-p_2)} \) for reasonable values of \( n \), might be better.  Now some
experimentation, including binary search, is necessary to choose a value
of \( n \ge 73 \) required to determine the biased coin with certainty \(
0.95 \).  The scripts below show the results of that experimentation and
compare it to the other determination methods based on the majority and
the Central Limit Theorem.  For example, with \( p_1 = 0.5 \) in the
left hand, and \( p_2 = 0.6 \) in the right hand it takes \( 189 \)
flips to indentify the fair coin in the left hand with \( 95\% \)
certainty.  A close look at the results show why the log-likelihood
Bayesian criterion requires more flips for the same certainty.  The
short answer is that the log-likelihood is a stricter requirement than
the majority requirement in the other two approaches.  In one trial with
\( 200 \) flips, the fair coin in the left hand came with \( 108 \)
heads while the biased coin in the right hand came up with \( 110 \)
heads.  The majority of heads was in the right hand, but the absolute
difference of the log-likelihoods is \( 0.8109302 \), not greater than \(
\log(1/\alpha-1) = 2.944439 \).  Both heads totals are well within the
two-standard deviation range around the respective expected values of \(
100 \) heads and \( 120 \) heads respectively.

An R script testing all three methods against placing the fair and
biased coins in the left and right randomly is in the scripts section.
In summary, even with more flips, the Likelihood method is slightly less
effective than the two majority methods.  It is also a more complicated
procedure.

\subsection*{Analysis of All Possible Algorithms}

The Weak Law of Large Numbers says the probability the biased coin will
have a majority of heads increases to \( 1 \) as the number of flips
increases.  Thus using an unlimited number of tosses will determine the
biased coin.  But efficiency asks for a necessary number of tosses to
determine the biased coin with a satisfactory confidence.  The
discussion above gives several algorithms each with some number of
tosses, often a large number, required to determine the biased coin with
some confidence.  But the question remains, are these the best
algorithms?  That is, among all possible algorithms is there a minimum
number of tosses necessary to determine the biased coin with a certain
confidence?  The goal of this subsection is to get a theoretical lower
bound on the necessary number of tosses.

Let \( X_1, X_2, X_3, \ldots \) be a sequence of i.i.d.\ Bernoulli
random variables with \( p = 1/2 \) and let \( Y_1, Y_2, Y_3, \ldots \)
be an independent sequence of i.i.d.\ Bernoulli random variables with \(
p = 1/2 + \epsilon \). Let \( 0 < \alpha < 1/2 \) be the error tolerance
or equivalently \( 1/2 < 1- \alpha < 1 \) is the confidence.  Let \(
\mathcal{A} \) be an algorithm, that is, a function which takes a binary
string \( Z = (Z_1, Z_2, Z_3, \ldots, Z_n) \) as input and outputs \( 0 \)
if the algorithm determines \( Z_i \sim X_i \) and outputs \( 1 \) if
the algorithm determines \( Z_i \sim Y_i \).  Let \( \bbP^m \) be the
distribution of \( X_1, X_2, X_3, \ldots, X_m \) and \( \bbQ^{m} \) be
the distribution of \( Y_1, Y_2, Y_3, \ldots, Y_M \).  The goal is to
find a value \( m \) so that \( \bbP
\left[ \mathcal{A}(Z)  = 0 \right] \ge 1-\alpha \) and \( \bbQ
\left[ \mathcal{A}(Z) = 1 \right] \ge 1-\alpha \).  In words, find the
value of \( m \) necessary to make the \( \bbP^m \)-measure of \(
\mathcal{A}(Z) = 0 \) large and simultaneously make the \( \bbQ^m \)-measure
of \( \mathcal{A}(Z) = 1 \) large.

Recall the \defn{Total Variation distance} of \( \bbP^m \) from \( \bbQ^
{m} \) is%
\index{total variation distance}
\[
    \| \bbP^m - \bbQ^m \|_{TV} = \max_{A \subset \Omega} \abs{ \bbP^m(A)
    - \bbQ^m(A)}.
\] So now the goal is to find a value \( m \) so that \( \| \bbP^m -
\bbQ^m \|_{TV} \ge 1 - 2\alpha \). Pinsker's Inequality says that for
measures \( \mu \) and \( \nu \)%
\index{Pinsker's Inequality}
\[
    D_{KL}(\mu \parallel \nu) \ge 2 \left( \| \mu - \nu \|_{TV} \right)^2
\] where \( D_{KL}(\mu \parallel \nu) \) is the Kullback-Liebler
Divergence of \( \mu \) compared to \( \nu \).%
\index{Kullback-Liebler Divergence}%
.  The Kullback-Liebler Divergence satisfies the identity \( D_{KL}(\mu^m
\parallel\nu^m) = m D_{KL}(\mu \parallel\nu) \), sometimes called the
``chain rule''.

Applying these notions to the case of distinguishing the coins,
\[
    m D_{KL}(\bbP \parallel \bbQ) = D_{KL}(\bbP^m \parallel \bbQ^m) \ge
    \| \bbP^m - \bbQ^m \|_{TV} \ge 2 (1 - 2\delta)^2.
\] Solving for \( m \)
\[
    m \ge \frac{2 (1 - 2\delta)^2}{D_{KL}(1/2 \parallel 1/2 + \epsilon)}
\] where \( D_{KL}(1/2 \parallel 1/2 + \epsilon) \) is a memorable
shorthand for \( D_{KL}(\bbP \parallel \bbQ) \).

Now expand the definition of \( D_{KL}(1/2 \parallel 1/2 + \epsilon) \):
\begin{align*}
    &= -\frac{1}{2} \left( \log \left( 1-2\epsilon \right) + \log \left(
    1+2\epsilon \right) \right) \\
    &= -\frac{1}{2} \log( 1- 4\epsilon^2 ) \\
    &= -\frac{1}{2} \cdot -4 \epsilon^2 = 2 \epsilon^{2}
\end{align*}
(Check this again, since it is better than the notes.)

Now combining this estimate with the inequality for \( m \)
\[
    m \ge \frac{2 (1 - 2\delta)^2}{2 \epsilon^2} = (1 - 2\delta)^2
    \epsilon^{-2}.
\] The requirement that \( \delta < 1/4 \) guarantees that \( (1 - 2\delta)^2
> 1/4 \).  However the main conclusion is that the lower bound for the
number of flips is of the order of \( \epsilon^{-2} \).

The next step demonstrates an algorithm that achieves this bound, up to
multiplicative constants. The proof uses the Chernoff bounds to put an
upper bound on \( m \) of the same order.  This means that there exists
an algorithm \( \mathcal{A} \) which achieves the \( \delta \) error
tolerance given \( m \) flips where \( m \) is of the order \( \epsilon^
{-2} \).

The Chernoff bound~%
\index{Chernoff bound}
for a binomial random variable \( W \sim
\operatorname{Bin}
(n,p) \) and \( 0 < \alpha < p \), then
\[
    \Prob{W < n\alpha} \le \EulerE^{-n D_{KL}(1/2 \parallel 1/2+\epsilon)
    }.
\] The Chernoff bound applied to a binomial random variable is the same
as the Large Deviation Bound~%
\index{large deviations}
for a binomial random variable, see the definition of \( h_+(\epsilon) \)
and the statement of Theorem 3 in \link{http://www.math.unl.edu/~sdunbar1/ProbabilityTheory/Lessons/BernoulliTrials/LargeDeviations/largedeviations.html}
{Large Deviations} in the section on Large Deviations.

Define algorithm \( \mathcal{A} \) taking \( Z = Z_1, Z_2, \ldots, Z_m \)
as input by
\[
    \mathcal{A}(Z) =
    \begin{cases}
        1 & \text{ (fair coin) if the numer of Heads is less than \(
        \frac{1}{2} + \frac{\epsilon}{2} \) } \\
        0 & \text{ (biased coin) if the numer of Heads is greater than \(
        \frac{1}{2} + \frac{\epsilon}{2} \) }.
    \end{cases}
\] By the Chernoff bounds
\begin{align*}
    \Prob{X_1 + X_2 + \cdots + X_m \le (\frac{1}{2} + \frac{\epsilon}{2})}
    &\le \EulerE^{-m D_{KL}(\frac{1}{2} + \frac{\epsilon}{2} \parallel
    \frac{1}{2}) } \\
    \Prob{Y_1 + Y_2 + \cdots + Y_m \le (\frac{1}{2} + \frac{\epsilon}{2})}
    &\le \EulerE^{-m D_{KL}(\frac{1}{2} + \frac{\epsilon}{2} \parallel
    \frac{1}{2}) }.
\end{align*}
The goal is to have both of these bounds less than \( \alpha \), so
it resolves to solving both inequalities for \( m \). The first
inequality becomes
\[
    m \ge \frac{\log(1/\alpha)}{D_{KL}(\frac{1}{2} + \frac{\epsilon}{2}
    \parallel \frac{1}{2})}.
\] Likewise the second inequality also becomes
\[
    m \ge \frac{\log(1/\alpha)}{D_{KL}(\frac{1}{2} + \frac{\epsilon}{2}
    \parallel \frac{1}{2})}.
\] As shown above \( D_{KL}(\frac{1}{2} + \frac{\epsilon}{2} \parallel
\frac{1}{2}) \le 2 \epsilon^{-2} \) so choosing \( m \) at least \( 2
\log(1/\alpha) \epsilon^{-2} \) suffices.

Note the difference in the bounds for \( \epsilon \) fixed.  The lower
bound grows like \( (1-2\alpha)^2 \) while the upper bound grows like \(
\log(1/\alpha) \), For the specific examples considered at the
beginning, \( \alpha = 0.05 \) and \( \epsilon = 0.10 \), the lower
bound is
\[
    (1 - 2\delta)^2 \epsilon^{-2} = (1- 2 \cdot 0.05)^2 \cdot (0.10)^{-2}
    = 81
\] and the upper bound is
\[
    2 \log(1/\alpha) \epsilon^{-2} = 2 \log(1/0.05) (0.10)^{-2} \approx
    599.15.
\] The algorithms described above fall well within this range, closer to
the lower bound.

\subsection*{Sources}

The definition of biased coin is from Wikipedia.

The simple probability problem is adapted from \link{https://www.cut-the-knot.org/Probability/TwoCoins.shtml}
{Bogolmony, Cut-the-Knot, Two Coins, One Fair, One Biased}.  The same
problem appears in \link{http://www.flyingcoloursmaths.co.uk/two-coins-one-fair-one-biased}
{FlyingColoursMaths Two Coins, One Fair, One Biased}.

The problem of how many tosses to find the biased coin appeared in the
FiveThirtyEight.com weekly Riddler puzzle column on September 29, 2017.
This first subsection with the majority solution is adapted from a white
paper ``What's Past is \emph{Not} Prologue'' by James White, Jeff
Rosenbluth, and Victor Haghani from Elm Partners Investing,
\cite{white17}. In turn, their example is motivated by a paper ``Good
and bad properties of the Kelly criterion'' by McLean, Thorp and
Ziemba,
\cite{mclean17}.  The second subsection with the statistical solution is
original.

The Bayesian analysis is adapted and extended from \\
\link{https://laurentlessard.com/bookproofs/finding-the-doctored-coin}{Lessard,Finding
the biased coin}.

The necessary bounds for an arbitrary algorithm are adapted from lecture
notes by Austin Eide.

\nocite{white17}
\nocite{maclean10}

\hr

\visual{Algorithms, Scripts, Simulations}{../../../../CommonInformation/Lessons/computer.png}
\section*{Algorithms, Scripts, Simulations}

\subsection*{Algorithm}

\input{twocoins_algorithms}

\subsection*{Scripts}

\input{twocoins_scripts} \hr

\visual{Problems to Work}{../../../../CommonInformation/Lessons/solveproblems.png}
\section*{Problems to Work for Understanding}
\renewcommand{\theexerciseseries}{}
\renewcommand{\theexercise}{\arabic{exercise}}

\begin{exercise}
    You have two coins.  One is fair with \( \Prob{H} = \frac{1}{2} \).
    The other coin is biased with \( \Prob{H} = p > \frac{1}{2} \).
    First you toss one of the coins once, resulting in heads.  Then you
    toss the other coin three times, resulting in two heads. What is the
    least value of \( p \) that will allow you to decide which is the
    biased coin based on this information?
\end{exercise}
\begin{solution}
    Interestingly, the probability of two heads in three tosses from a
    fair coin and one head in one toss from a biased coin with \( p >
    1/2 \), specifically \( \frac{3p}{8} \), is always greater, for any \(
    p \), than the probability of two heads in three tosses from a
    biased coin and one head in one toss from a fair coin, specifically \(
    \frac{3}{2}p^2(1-p) \).
\end{solution}
\begin{exercise}
    Show that the log-likelihood expression with the binomial
    probabilities simplifies to
    \[
        \abs{k_L - k_R} \cdot \abs{ \log\left( \frac{1}{p_1} - 1 \right)
        - \log\left( \frac{1}{p_2} - 1 \right) } > \log \left( \frac{1}{\alpha}
        - 1 \right).
    \]
\end{exercise}
\begin{solution}
    Start with
    \[
        \frac{Q(k \given \theta_L)}{Q(k \given \theta_R)} = \left(\frac{p_1}
        {p_2} \right)^{k_L - k_R} \cdot \left(\frac{1-p_2}{1-p_1} \right)^
        {k_L - k_R}.
    \] Note that the binomial coefficients are the same in numerator and
    denominator so they cancel. Take logarithms of both sides, use laws
    of exponents
    \[
        \log(Q(k \given \theta_L)) - \log(Q(k \given \theta_R)) = (k_L -
        k_R) \log \left( \left(\frac{p_1}{p_2} \right) + (k_L - k_R)
        \log \left(\frac{1-p_2}{1-p_1} \right) \right).
    \] Factor and rearrange
    \[
        \log(Q(k \given \theta_L)) - \log(Q(k \given \theta_R)) = (k_L -
        k_R) \left( \log(p_1) - \log(p_2) + \log(1-p_2) - \log(1-p_1)
        \right).
    \] Rearrange again
    \begin{align*}
        \log(Q(k \given \theta_L)) - \log(Q(k \given \theta_R)) &= (k_L
        - k_R) \left( \log \left( \frac{p_1}{1-p1} \right) - \log \left(\frac
        {p_{2}}{1-p_2}\right) \right) \\
        & = (k_L - k_R) \left( \log \left( -\frac{1-p1}{p_1} \right) +
        \log \left(\frac{1-p_2}{p_{2}}\right) \right) \\
    \end{align*}
    Expanding the fractions and taking absolute values gives the desired
    right hand side
    \[
        \abs{k_L - k_R} \cdot \abs{ \log\left( \frac{1}{p_1} - 1 \right)
        - \log\left( \frac{1}{p_2} - 1 \right) } > \log \left( \frac{1}{\alpha}
        - 1 \right).
    \]
\end{solution}

\hr

\visual{Books}{../../../../CommonInformation/Lessons/books.png}
\section*{Reading Suggestion:}

\bibliography{../../../../CommonInformation/bibliography}

%   \begin{enumerate}
%     \item
%     \item
%     \item
%   \end{enumerate}

\hr

\visual{Links}{../../../../CommonInformation/Lessons/chainlink.png}
\section*{Outside Readings and Links:}
\begin{enumerate}
    \item
    \item
    \item
    \item
\end{enumerate}

\section*{\solutionsname}
\loadSolutions

\hr

\mydisclaim \myfooter

Last modified:  \flastmod

\end{document}

File name                  : twocoins.tex
Number of characters       : 42662
Number of words            : 5260
Percent of complex words   : 18.23
Average syllables per word : 1.7202
Number of sentences        : 278
Average words per sentence : 18.9209
Number of text lines       : 825
Number of blank lines      : 196
Number of paragraphs       : 185


READABILITY INDICES

Fog                        : 14.8611
Flesch                     : 42.1055
Flesch-Kincaid             : 12.0869


